\if 0
% here's a way to try to make it just be one column wide but it didn't
% work for me, it made it one columnwide but it moved it to the end of the
%document
\begin{wrapfigure}{r}{0.5\textwidth}[tb]
    \begin{center}
        \psfig{file=figs/architecture.eps,width=0.48\textwidth}
    \end{center}
% we can also try to use SCfigure for captions on the side, I tried that
% and it looks sort of weird.  But I'm also not happy with the default
% which makes it way too big and leaves too much white space..
\begin{SCfigure}
    \centering
    \includegraphics[width=0.5\textwidth]{figs/architecture.eps}
\fi

\singlespace
\begin{figure}[tb]
    \centering
    \includegraphics[width=0.7\textwidth]{figs/architecture.eps}
    \mycaption{fig-arch}{Data Reorganization.}{
This figure depicts how PLFS reorganizes an N-1 strided checkpoint file onto
the \upfs.  A parallel application consisting of six
processes on three compute nodes is represented by the top three boxes.  Each 
box represents a compute node, a circle is a process, and the three small boxes
below each process represent the state of that process. The processes
create a new file on \plfs\ called \textit{checkpoint1}, causing  \plfs\ in
turn to
create a \Term{container} structure on the underlying parallel file system.
The container consists of a top-level directory also called
\textit{checkpoint1} and several
sub-directories to store the application's data.  For each
process opening the file, \plfs\ creates a \Term{data file} within one of the
sub-directories, it also creates one \Term{index file} within that same 
sub-directory 
which is shared by all processes on a compute node.
For each write, \plfs\ appends the data to the corresponding data file 
and appends a record into the appropriate index file.  This record
contains the length of the write, its logical offset, and a pointer
to its physical offset within the data file to which it was appended.
To satisfy reads, \plfs\ aggregates these index files to create a lookup
table for the logical file.  Also shown in this figure are the
\textit{access} file, which is used to store ownership and privilege
information about the logical file, and the \textit{openhosts} and
\textit{metadata} sub-directories which are used to cache metadata in order
to improve query time (\eg\ a \syscall{stat} call). 
}
\end{figure}
\doublespace
