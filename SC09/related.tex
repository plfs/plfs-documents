\section{Related Work}
\label{related}

% Remzi says refer to Erez Zadok's work on stackable file systems
% also says to refer to Thain's interposition work.  But we already did.

%Sudharshan Vazhkudai has several papers on checkpointing, typically using
%memory or scavenged local disk as a buffer.  Descriptions available at his
%webpage: \url{http://www.csm.ornl.gov/~vazhkuda/}.

%Frank Shorter's Master Thesis at Clemson is a big discussion about parallel
%benchmarks: \url{ftp://ftp.parl.clemson.edu/pub/techreports/2003/PARL-2003-001.ps}.

Translating random access checkpoint writes into a sequential pattern is an
idea which extends naturally from work on log-structured file
systems~\cite{lfs} such as NetApp's WAFL~\cite{WAFL} and Panasas's Object
Storage~\cite{welch08scalableperformance}. While these ideas reorganize disk
layout for sequential writing, they do not decouple concurrency caused by
multiple processes writing to a single file.  Another approach to
log-structuring N-1 patterns addressed only physical layout and also did not
decouple concurrency~\cite{pvfs-log}.  We believe our contribution of
rearranging N-1 checkpoints into N-N is a major advance. 

Checkpoint-restart is an old and well studied fault tolerance strategy. A broad
evaluation of rollback-recovery protocols and checkpointing and a discussion of
their future at the petascale can be found in \cite{survey1, survey2}.  

Berkeley Lab Checkpoint/Restart~\cite{blcr} and Condor
Checkpointing~\cite{condor-ckpt} both allow unmodified applications to
checkpoint the state of a single node.  They leverage the operating system's
ability to swap a process to save the totality of the swap image persistently.
The great strength of this approach is that it can be used by applications that
have no internal checkpoint mechanism.  A disadvantage is that these
checkpoints are larger then when an application specifies only the exact data
that it needs saved.
Conversely, incremental checkpointing and memory
exclusion~\cite{pcl:99:me,pxn:95:scd} reduce the size of checkpoint data by
only saving data that has changed since the previous checkpoint.  Buffering
with copy-on-write~\cite{li-298215} can also reduce checkpoint latency.

%allows user applications to checkpoint their state, including system
%resources, without the application programmer having to explicitly code the
%process. In this regard, it has a similar philosophy to \plfs\ of making
%checkpointing easier for the application developer, but our focus is on
%enabling better checkpointing performance for existing applications rather than
%writing an entire checkpointing system.  

%\scribble{Couldn't find the incremental checkpointing reference}


\newcommand{\C}[1]{\centering{#1}}
%\newcommand{\C}[1]{\begin{center}#1\end{center}}
\hyphenation{Immediately}
\begin{table*}
\begin{center}
\begin{tabular}{|r|p{19mm}|p{20mm}|p{23mm}|p{15mm}|p{19mm}| p{19mm}|}
\hline 
{\em } & 
\C{{\em Interposition Technique Used}} & 
\C{{\em No Extra Resources Used During}} & 
\C{{\em No Extra Resources Used After}} & 
\C{{\em Maintains Logical Format}} & 
\C{{\em Works with Unmodified Applications}} & 
\C{{\em Data Immediately Available}} 
\tabularnewline \hline
ADIOS & \C{Library} & \C{Yes} & \C{Yes} & \C{Yes} &  \C{No} & \C{Yes}
\tabularnewline \hline
{\tt stdchk} & \C{FUSE} & \C{No (LD, M)} & \C{No (LD, N, M)} & \C{Yes} & \C{Yes} & \C{Yes}
\tabularnewline \hline
%Neighbor &  \C{FUSE} &\C{No (M)} & \C{No (M, N)} & \C{Yes} & \C{Yes} & \C{Yes} 
%\tabularnewline \hline
Diskless &  \C{Library} &\C{No (M)} & \C{No (M)} & \C{No} & \C{No} & \C{Yes} 
\tabularnewline \hline
Sp Writ &  \C{Library} &\C{Yes} & \C{Yes} & \C{Yes} & \C{No} & \C{No} 
\tabularnewline \hline
ZEST &  \C{FUSE} &\C{No (RD)} & \C{No (RD)} & \C{No} & \C{No} & \C{No} 
\tabularnewline \hline
PLFS & \C{FUSE} & \C{Yes} & \C{Yes} & \C{Yes} & \C{Yes} & \C{Yes} 
\tabularnewline \hline
\end{tabular}
\mycaption{tab-feature}{Techniques for improving N-1 Checkpointing}{
This table presents a comparison of the various techniques for reducing
N-1 checkpoint times.  For exposition, we have used various
shorthands: 
%Neighbor for Neighbor's Memory,
Diskless for Diskless 
Checkpointing, Sp Writ for Lustre Split Writing, LD for local disk on the compute nodes, RD for remote disk 
on the storage system, M for memory, and N for network.  
}
\end{center}
\end{table*}

\if 0
other possible columns:

            Speed
    ADIOS: Disk                                                                 
    stdck: Disk                                                                 
    Neighbor: Network                                                           
    Diskless: Network                                                           
    ZEST: Spindle                                                               
    PLFS: N-N  

lines of code, user or kernel mode, interposition technique, maintainability,
portability

also if we add more columns we'll have to rotate it so columns become rows and
rows become columns.

\fi


%A variant of N-N checkpoint uses the disks or memory of computer nodes to
%record a checkpoint writing it to storage as a background process during the
%next compute period. 

{\tt stdchk}~\cite{stdchk} saves checkpoints into a cloud of free disk space
scavenged from a cluster of workstations.  A similar
approach~\cite{aggregate-memory} reserves compute node memory to temporarily
buffer checkpoints and then asynchronously saves them to persistent storage.
Diskless checkpointing~\cite{diskless} also  
saves checkpoints into compute node memory, but does not subsequently transfer
them to persistent storage.  Rather it achieves fault protection by using erasure 
coding on the distributed checkpoint image. 
Although these techniques work well for many applications, large HPC parallel
applications jealously utilize all memory and demand a high degree of
determinism in order to avoid jitter~\cite{DusseauPhd98} and are therefore seen
to be poorly served by techniques reducing available memory or techniques which
require background processes running during computation.

The Adaptable IO System (ADIOS)~\cite{adios} developed by the Georgia Institute
of Technology and Oak Ridge National Laboratory provides a high-level IO API
that can be used in place of HDF5 to do much more aggressive write-behind and
log-like reordering of data location within the checkpoint. While this technique
requires application modification, it enables interoperability with other
middleware libraries.  Similarly, Lustre Split Writing~\cite{lsw}, uses a 
library approach and leverages Lustre's file joining mechanism to decouple
concurrent access at runtime as does \plfs.  However, Lustre Split Writing is
wed to the Lustre file system, requires application modification, and prevents
the data from being immediately available following a checkpoint.

ZEST~\cite{zest}, developed at Pittsburgh Supercomputing Center, is a file
system infrastructure that is perhaps the most similar in philosophy to \plfs,
particularly in its borrowing of concepts from log-structured file systems.
Rather than each client process pushing their writes sequentially to storage,
in ZEST manager threads running on behalf of each disk pull data from a series
of distributed queues, in a technique borrowed from
River~\cite{Arpaci-Dusseau03-TOCS}.  The key aspect of ZEST is that no
individual write request is ever assigned to a specific disk; disks pull from
these queues whenever they are not busy, resulting in high spindle utilization
even in a system where some devices are performing more slowly than others.
Unlike \plfs, however, data is not immediately available to be read, requiring
a subsequent phase to first rewrite the file before it can be accessed.  Since
this phase happens in the relatively long time between checkpoints and since it
happens on the server nodes and not on the compute nodes, this subsequent
rewrite phase is not typically problematic for a dedicated checkpoint file
system.
