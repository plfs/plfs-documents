\begin{figure}[tb]
    \centering
        \readgraph{Uniform Restart}{nn}{data/reads/nn.eps}
        \readgraph{Non-uniform Restart}{nm}{data/reads/nm.eps}
        \readgraph{Archive Copy}{n1}{data/reads/archive.eps}
    \\
    \mycaption{fig-reading}{Read Bandwidth.}{
These three graphs show the results of our read measurements on the
\rrz. We created a set 20~\GB\ N-1 checkpoint files through
\plfs\ and another directly on PanFS.  Each file was produced by a different
number of writers; all of the \syscall{writes} were 47001 bytes in size.  For
each graph, the \yaxis\ shows the read bandwidth as a function of the number of
writers who created the file.  The graph on the left shows the read bandwidth
when the number of readers is the same as the number of writers, as is the case
in a typical uniform restart; in this case, the size of the reads is the same
as the size of the original writes.  The graph in the middle emulates a
non-uniform restart in which the application resumes on one fewer compute
nodes; in this case, the size of the \syscall{reads} is slightly larger than
the size of the original \syscall{writes}.  Finally, the graph on the right
shows the read bandwidth when there only four readers; we used LANL's archive
copy utility and modelled the common scenario of copying checkpoint data to an
archive system using a relatively small number of readers.  To enable
comparison across graphs, the axis ranges are consistent.
} 

\end{figure}

