\section{Current Status and Future Work}
\label{future}

\scribble{I removed the 'cute' part of this. Sorry, john}
We initially developed \plfs\ in order to test our hypothesis that an
interposition layer could rearrange checkpoint patterns such that the
convenience of N-1 patterns could be preserved while achieving the bandwidth of
N-N.  However, after achieving such large improvements with real LANL
applications, we were compelled to harden it 
into a production file system. Consisting of about three thousand lines of
code, \plfs\ is currently a permanent mount point on Roadrunner where several
applications have begun evaluating whether it is appropriate for their write 
patterns.

%Since \plfs\ consists of only three thousand
%source lines of code and is easily ported to all Linux architectures, we
%decided~\footnotemark that the costs of maintaining it are easily outweighed by
%the benefits.
%\footnotetext{were persuaded by management}

Although \plfs\ works transparently with unmodified applications, in practice
some applications may need to make minor changes.  Applications that open
\plfs\ files in read-write mode will find that reading from these files can be
very slow.  As we discussed in Section~\ref{arch-read}, reads in \plfs\ are
handled by reading and aggregating the index files.  When files are opened in
write-read mode, this process must be repeated for every \syscall{read} since
intervening \syscall{writes} may have occurred.  Although the vast majority of
HPC applications are able to read in read-only mode, we do plan to investigate
removing this limitation in the future for those few applications who must occasionally
read while writing.  One possibility is to introduce metadata
servers which can synchronously maintain an aggregated index in memory. 

Originally intended merely to address N-1 challenges, \plfs\, as currently
designed, also has the potential to address N-N challenges.  One way that
\plfs\ can reduce N-N checkpointing times is by reducing disk seeks through its
use of log-structured writing.  However, we have yet to measure the
frequency of non-sequential IO within N-N checkpoints.

\scribble{Added gigaplus reference to the end. --Milo}
Another challenge of an N-N pattern is the
overwhelming metadata pressure resulting from the simultaneous creation of tens
of thousands of files within a single directory.  Currently HPC parallel file
systems do distribute metadata across multiple metadata servers; however they do so
at the granularity of a \Term{volume} or a directory (\ie\ all files within a
directory share a single metadata server).  \plfs\ can easily refine this
granularity by distributing container subdirectories 
across multiple metadata servers.  At the moment, \plfs\ only creates container
structures for regular files; directories are simply created directly on the
\upfs.  By extending \plfs\ to create a similar container structure for
directories, we believe that \plfs\ can effectively address this N-N challenge
as well, in a manner similar to~\cite{gigaplus}.
