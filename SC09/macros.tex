\newcommand{\plfs}{PLFS}
\newcommand{\noopfs}{No-opFS}
\newcommand{\TITLE}{\plfs: A Checkpoint Filesystem for Parallel Applications}
\newcommand{\Title}[1]{{\bf #1}}
\newcommand{\Term}[1]{{\em #1}}
\newcommand{\eg}{\textit{e.g.}}
\newcommand{\ie}{\textit{i.e.}}
\newcommand{\etal}{\textit{et al.}}
\newcommand{\Math}[1]{$#1$}
\newcommand{\RAID}{RAID}
\newcommand{\url}[1]{\textit{#1}}
\newcommand{\syscall}[1]{\textit{#1}}
\newcommand{\upfs}{underlying parallel file system}
\newcommand{\rrz}{Roadrunner test cluster}

\newcommand{\KB}{~KB}
\newcommand{\KBs}{~KB/s}
\newcommand{\Kbs}{~Kbit/s}
\newcommand{\mbs}{~Mbit/s}
\newcommand{\MB}{~MB}
\newcommand{\GB}{~GB}
\newcommand{\MBs}{~MB/s}
\newcommand{\GBs}{~GB/s}
\newcommand{\mus}{\mbox{$\mu s$}}
\newcommand{\ms}{\mbox{$ms$}}

\newcommand{\xaxis}{x-axis}
\newcommand{\yaxis}{y-axis}

\newlength{\captwidth} \setlength{\captwidth}{0.45\textwidth}
\newlength{\captsize}  \let\captsize=\footnotesize

\newcommand{\motivationgraph}[2]{
        \subfloat[#1]{\label{fig:#2-motivation}
           \includegraphics[width=0.25\textwidth]{data/#2/#2-motivation.eps}}
}

\newcommand{\readgraph}[3]{
        \subfloat[#1]{\label{fig:#2-read}
           \includegraphics[width=0.25\textwidth]{#3}}
}

\newcommand{\tablefigure}[2]{
    \subfloat[#1]{\label{eval:#2}\includegraphics[width=0.25\textwidth]
        {data/#2/#2.eps}}
}

% works and makes it smaller but it doesn't put it at the bottom of the page
\newcommand{\myfootnotetext}[1]{
    {\begin{minipage}%{\captwidth}
    \let\normalsize=\captsize
    \footnotetext{#1}
    \end{minipage}}
}

%\newcommand{\beforecaption}{\vspace{-.15cm}\begin{spacing}{0.85}}
%\newcommand{\aftercaption}{\hrulefill\vspace{-0.45cm}\end{spacing}}
% \newcommand{\aftercaption}{\vspace{-.45cm}\end{spacing}}
\newcommand{\beforecaption}{\vspace{.2cm}}
\newcommand{\aftercaption}{\vspace{-.2cm}}
\newcommand{\mycaption}[3]{
            \beforecaption
            \caption[#2]{\label{#1}{{\small \bf #2}}
            {{\small \em #3}\aftercaption}}
}

%\newcommand{\scribble}[1]{{{\marginpar{\tiny#1}}}}
\newcommand{\scribble}[1]{}
