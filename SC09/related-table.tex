\newcommand{\C}[1]{\centering{#1}}
%\newcommand{\C}[1]{\begin{center}#1\end{center}}
\hyphenation{Immediately}
\begin{table}
\begin{center}
\begin{tabular}{|r|p{19mm}|p{20mm}|p{23mm}|p{15mm}|p{19mm}| p{19mm}|}
\hline 
{\em } & 
\C{{\em Interposition Technique Used}} & 
\C{{\em No Extra Resources Used During}} & 
\C{{\em No Extra Resources Used After}} & 
\C{{\em Maintains Logical Format}} & 
\C{{\em Works with Unmodified Applications}} & 
\C{{\em Data Immediately Available}} 
\tabularnewline \hline
ADIOS & \C{Library} & \C{Yes} & \C{Yes} & \C{Yes} &  \C{No} & \C{Yes}
\tabularnewline \hline
{\tt stdchk} & \C{FUSE} & \C{No (LD, M)} & \C{No (LD, N, M)} & \C{Yes} & \C{Yes} & \C{Yes}
\tabularnewline \hline
%Neighbor &  \C{FUSE} &\C{No (M)} & \C{No (M, N)} & \C{Yes} & \C{Yes} & \C{Yes} 
%\tabularnewline \hline
Diskless &  \C{Library} &\C{No (M)} & \C{No (M)} & \C{No} & \C{No} & \C{Yes} 
\tabularnewline \hline
ZEST &  \C{FUSE} &\C{No (RD)} & \C{No (RD)} & \C{No} & \C{No} & \C{No} 
\tabularnewline \hline
PLFS & \C{FUSE} & \C{Yes} & \C{Yes} & \C{Yes} & \C{Yes} & \C{Yes} 
\tabularnewline \hline
\end{tabular}
\mycaption{tab-feature}{Techniques for improving N-1 Checkpointing}{
This table presents a comparison of the various techniques for reducing
N-1 checkpoint times.  For exposition, we have used various
shorthands: 
%Neighbor for Neighbor's Memory,
Diskless for Diskless 
Checkpointing, LD for local disk on the compute nodes, RD for remote disk 
on the storage system, M for memory, and N for network.  
}
\end{center}
\end{table}

\if 0
other possible columns:

            Speed
    ADIOS: Disk                                                                 
    stdck: Disk                                                                 
    Neighbor: Network                                                           
    Diskless: Network                                                           
    ZEST: Spindle                                                               
    PLFS: N-N  

lines of code, user or kernel mode, interposition technique, maintainability,
portability

also if we add more columns we'll have to rotate it so columns become rows and
rows become columns.

\fi
