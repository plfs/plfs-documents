% eval figures all in one big table
\begin{figure}[t!]
\centering
\begin{tabular}{ccc}
    \tablefigure{MPI-IO Test on PanFS}{panfs}
    &
    \tablefigure{MPI-IO Test on GPFS}{gpfs}
    &
    \tablefigure{MPI-IO Test on Lustre}{lustre}
    \\
    \tablefigure{LANL Anonymous 1}{lanl1}
    &
    \tablefigure{LANL Anonymous 2}{lanl2}
    &
    \tablefigure{LANL Anonymous 3}{lanl3}
    \\
    \tablefigure{PatternIO}{pattern}
    &
    \tablefigure{QCD}{qcd}
    &
    \tablefigure{BTIO}{btio}
    \\
    \tablefigure{FLASH IO}{flash}
    &
    \tablefigure{SUMMARY}{summary}
    &
    \tablefigure{Chombo IO}{chombo}
\end{tabular}
\mycaption{fig-eval}{Experimental Results.}{
The three graphs in the top row are the same graphs that were presented
earlier in Figure~\ref{fig-motivation}, except now they have an additional
line showing how PLFS allows an N-1 checkpoint to achieve most, if not all,
of the bandwidth available to an N-N checkpoint. 
The bar graph in the center of the bottom row consolidates these results and
shows a pair of bars for each, showing both the relative minimum and the
maximum speedups achieved across the set of experiments.  Due to radically
different configurations for these various experiments, the axes for 
these graphs are not consistent.  The relative comparison within each graph
should be obvious; absolute values can be ascertained by reading the axes.
}
\end{figure}
