\section{Background}
\label{motivation}

% Why do we care about checkpointing
\if 0
As a result, applications 
periodically save
the current state of their computation to persistent storage.  Later, after a
failure is encountered, the application can restart from the last saved
checkpoint. 
\fi

% N-N compared to N-1

\scribble{Lots of tweaking of final sentence} 
From a file system perspective,
there are two basic checkpointing patterns: ~\Term{N-N} and ~\Term{N-1}.  An
N-N checkpoint is one in which each of ~\Math{N} processes writes to a unique
file, for a total of ~\Math{N} files written.  An N-1 checkpoint differs in
that all of ~\Math{N} processes write to a single shared file.  Applications
using N-N checkpoints usually write sequentially to each file, an access
pattern ideally suited to parallel file systems. Conversely, applications using
N-1 checkpoint files typically organize the collected state of all N processes
in some application specific, canonical order, often resulting in 
small, unaligned, interspersed writes.

Some N-1 checkpoint files are logically the equivalent of concatenating the
files of an N-N checkpoint (\ie\ each process has its own unique region
within the shared file).  This is referred to as an ~\Term{N-1 segmented}
checkpoint file and is extremely rare in practice.  More common is an
~\Term{N-1 strided} checkpoint file in which the processes write multiple small
regions at many different offsets within the file; these offsets are typically
not aligned with file system block boundaries~\cite{plfs-maps}.  N-1 strided
checkpointing applications often make roughly synchronous progress such that
all the processes tend to write to the same region within the file concurrently, and collectively this region sweeps across the file.
These three patterns, N-N, N-1 segmented, and N-1 strided, are
illustrated in Figure~\ref{fig-patterns}.  Since N-1 segmented is a rare pattern
in practice, hereafter we consider only N-1 strided and we refer to it with the
shorthand N-1. 

\scribble{Should probably explain or remove 'false sharing' in figure}
\scribble{If those Argonne graphs were decipherable in B-W, I'd like to include
two of them to visualize the difference between strided and non.}

% file system challenges for N-N and N-1
The file system challenge for an N-N workload is the concurrent creation of
thousands of files which are typically within a single directory.  An N-1
workload can be even more challenging however for several different reasons
which may depend on the particular parallel file system or the underlying
\RAID\ protection scheme.  In the Panasas ActiveScale parallel
file system~\cite{Welch04managingscalability} (PanFS), for example, small strided
writes within a parity stripe must serialize in order to maintain consistent
parity.  In the Lustre parallel file system~\cite{lustre}, writes to N-1
checkpoint files which are not aligned on file system block boundaries cause
serious performance slowdowns as well~\cite{nersc}.  Although both
N-N and N-1 patterns pose challenges, it has been our observation, as well
as that of many others~\cite{grider-hec,adios,nersc,zest,Thakur99datasieving},
that the challenges of N-1 checkpointing are more difficult. 
Applications using N-1 patterns consistently achieve significantly less
bandwidth than do those using an N-N pattern. 

\begin{figure}[tb]
    \centering
        \motivationgraph{PanFS}{panfs}
        \motivationgraph{GPFS}{gpfs}
        \motivationgraph{Lustre}{lustre}
    \\
    \mycaption{fig-motivation}{Motivation.}{
These three graphs demonstrate the large discrepancy between achievable
bandwidth and scalability using N-N and N-1 checkpoint patterns on three of 
the major HPC parallel file systems. 
    }
\end{figure}



Figure~\ref{fig-motivation} presents experimental data validating this
discrepancy: An N-1 checkpoint pattern receives only a small fraction of the
bandwidth achieved by an N-N pattern on PanFS, GPFS, and Lustre and does not
scale with increased numbers of nodes.  The PanFS experiment, run on LANL's
Roadrunner supercomputer using its 1000 blade PanFS storage system, shows a
maximum observed N-N bandwidth of 31~\GBs\ compared to a maximum observed N-1
bandwidth of less than 100~\MBs.  Although we show PanFS results using their
default RAID-5 configuration, PanFS has also a RAID-10 configuration which
reduces the implicit sharing caused by N-1 patterns when two writers both need
to update the same parity.  While this solution improves scaling and offers
much higher N-1 write bandwidth without sacrificing reliability, it does so by
writing every byte twice, a scheme that, at best, can achieve only
approximately half of the write bandwidth of N-N on RAID-5.  \plfs, however, as
will be shown in Section~\ref{eval}, can get much closer.

The GPFS and Lustre experiments were run on much smaller systems. The GPFS
experiment was run using an archive attached to Roadrunner using its 
nine, quad-core, file transfer nodes.  The Lustre experiment was run using five
client machines, each with eight cores, and twelve Lustre servers.  All three
file systems exhibit similar behavior; N-N bandwidths are consistently higher
than N-1 by at least an order of magnitude.  Measurements were gathered using
the LANL synthetic checkpoint tool, \Term{MPI-IO Test}~\cite{mpi-io-test}.
For each of these graphs, the size of each \syscall{write} was
47001 bytes (a small, unaligned number observed in actual applications to be
particularly problematic for file systems). \syscall{Writes} were
issued until two minutes had elapsed. Although this is atypical since 
applications tend to write a fixed amount of data instead of writing for
a fixed amount of time, we have observed that this allows representative 
bandwidth measurements with a predictable runtime. 

% why all applications not N-N
Since N-N checkpointing derives higher bandwidth than N-1, the obvious path to
faster checkpointing is for application developers to rewrite existing N-1
checkpointing applications to do N-N checkpointing instead.  Additionally, all
new applications should be written to take advantage of the higher bandwidth
available to N-N checkpointing.  Although some developers have gone this route,
many continue to prefer an N-1 pattern even though its disadvantages are well
understood.  There are several advantages to N-1 checkpointing that appeal to
parallel application developers.  One, a single file is much easier to manage
and to archive.  Two, N-1 files usually organize data into an application
specific canonical order that commonly aggregates related data together in
contiguous regions, making visualization of intermediate state simple and
efficient.  Additionally, following a failure, a restart on a different number
of compute nodes is easier to code as the checkpoint format is independent of
the number of processes that captured the checkpoint; conversely, gathering the
appropriate regions from multiple files or from multiple regions within a
single file is more complicated.  

% a list of some apps that do N-1

Essentially these developers have once again shifted complexity to the parallel
file system for their own convenience.  This is not unreasonable; it has long
been the province of computer systems to make computing more convenient for its
users.  Many important applications have made this choice.  Of the twenty-three
applications listed on the Parallel I/O Benchmarks page~\cite{pio-benchmarks},
at least ten have an N-1 pattern; two major applications at LANL use an N-1
checkpointing pattern as do at least two of the eight applications chosen to
run on Roadrunner during its initial stabilization phase.  N-1 checkpointing is
very important to these applications.  For example, at the core of one of
LANL's N-1 applications is a twenty-year old Fortran checkpointing library. 
About a decade ago, in response to a growing clamor about the
limitations of N-1 checkpointing bandwidth, developers for this application
augmented their checkpointing library with fifteen thousand lines of code.
However, instead of changing the application to write an N-N pattern, they
added a prefer to the IO routiones in which interprocess communication is used to aggregate and buffer writes.
Although they did not improve the performance to match that of other 
applications using an N-N checkpoint pattern, 
this effort was considered a success as they did improve the N-1
performance by a factor of two to three. This new checkpointing library,
called \Term{bulkio}, has been maintained over the past decade and ported to
each new successive supercomputer at LANL~\cite{bent-personal-bulk}. 
Furthermore, N-1 patterns continue to be developed anew in many new
applications.  High level data formatting libraries such as Parallel
NetCDF~\cite{pnetcdf} and HDF5~\cite{hdf5} offer convenience to application
developers who simply describes the logical format of their data and need no
longer consider how that data is physically organized in the file system.  Once
again this convenience merely shifts the complexity to the \upfs\ 
since these libraries use an N-1 pattern.


% Milo's change which I didn't like
%More generally, parallel application developers employ data formatting libraries such as NetCDF~\cite{pnetcdf} and HDF5~\cite{hdf5} to simplify programming, but these extra layers of mapping can lead to decreased locality and sequentiality when checkpointing a logically related set of variables. N-1 write access patterns can appear in new applications this way.
