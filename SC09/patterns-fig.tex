\begin{figure}[tb]
  \centering
    \fbox{
        \subfloat[N-N]{\label{fig:nn}
            \includegraphics[width=0.25\textwidth]{figs/nn-arrows.eps}}
    }
    \fbox{
        \subfloat[N-1 segmented]{\label{fig:n1segmented}
           \includegraphics[width=0.25\textwidth]{figs/n1nonstrided-arrows.eps}}
    }
    \fbox{
        \subfloat[N-1 strided]{\label{fig:n1strided}
            \includegraphics[width=0.25\textwidth]{figs/n1strided-arrows.eps}}
    }
    \mycaption{fig-patterns}{Common Checkpointing Patterns.}{
This figure shows the three basic checkpoint patterns: from left to right, N-N,
N-1 segmented, and N-1 strided.  In each pattern, the parallel application is
the same, consisting of six processes spread across three compute nodes each of
which has three blocks of state to checkpoint.  The difference in the three
patterns is how the application state is logically organized on disk.  In the
N-N pattern, each process saves its state to a unique file.  N-1 segmented is
simply the concatenation of the multiple N-N files into a single file.
Finally, N-1 strided, which also uses a single file as does N-1 segmented, has
a region for each block instead of a region for each process.  From a parallel
file system perspective, N-1 strided is the most challenging pattern as it is
the most likely to cause small, interspersed, and unaligned writes.  Note that
previous work~\cite{hedges-lunatic} refers to N-N as \Term{file per process}
and N-1 as \Term{shared file} but shares our segmented and strided terminology.
}
\end{figure}
