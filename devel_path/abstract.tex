\begin{abstract}
Experiment management requires a constant feedback loop cycling through
planning, execution, measurement, and analysis.  Experiment management in High
Performance Computing (HPC) follows this general pattern with three additional
characteristics. One, HPC applications must deal with frequent platform
failures which can interrupt, perturb, or terminate experiments. Two, these
applications typically use MPI.  Three, a scheduling system typically acts as a
gate-keeper.  This paper introduces \namefull\ (\name), an experimental
management framework for HPC which simplifies all four phases of experiment
management. Planning is simplified by allowing the user to merely describe
their experimental goals instead of constructing parameters for each individual
task. To simplify execution, \name\ dispatches the tasks itself thereby freeing
the user from remembering the often arcane methods for interacting with various
scheduling systems. \name\ provides transducers that automatically measure and
record important information; these can be extended to collect measurements
specific to each experiment.  Finally, analysis is simplified by providing 
\dbviz, a visualization tool enabling interactive data exploration. 
\end{abstract}
