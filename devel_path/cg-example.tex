\begin{figure*}[tb]
\subfloat[Experiment Description]{\label{mdtest-desc}
\lstinputlisting{examples/mdtest.py}}
\hspace{20pt}
\subfloat[Generated Commands]{\label{mdtest-commands}
\lstinputlisting{examples/mdtest.out}}
\mycaption{mdtest}{Command Generation.}{
On the left is a simple experiment description for the parallel
metadata benchmark \Term{mdtest}.  Lists are specified 
the \Term{-l} and the \Term{-z} options whereas just a single value
is specified for the \Term{-d} and \Term{-b} options.  The options list for 
\Term{mpi}
specifies that each generated command should be run at different sizes 
(passed to \Term{mpirun} with the \Term{-np} option. 
This example also shows how transducers can be 
specified to the \cg.  The right side shows the commands generated
from this experiment description.  Note that the user can either create
an experiment description file as in the left column or can manually create
a list of commands like those on the right side.  The \cg\ will determine
whether it needs to automatically generate commands from a description file
or whether it merely reads the commands directly.  Once the \cg\ either
generates the commands or reads them in, it then passes them to the \dispatcher.
}
\end{figure*}
