\documentclass[10pt]{article}
\usepackage{fullpage}

\usepackage{enumitem}
\setlist{noitemsep,topsep=0pt}
\setitemize{noitemsep,topsep=0pt}
\setenumerate{noitemsep,topsep=0pt} 

% macros to use in the text
\newcommand{\TITLE}{PLFS Roadmap}
\newcommand{\mds}{metadata distribution layer}
\newcommand{\fuse}{PLFS-FUSE}
\newcommand{\adio}{PLFS-ADIO}
\newcommand{\store}{back-end store}
\newcommand{\plfsrc}{\Path{/etc/plfsrc}}

%\newcommand{\summary}[1]{\framebox[6in]{#1}}

\newcommand{\compatibility}[1]{\subsection{Compatibility}{#1}}
\newcommand{\version}[2]{\subsection{Version #1 Summary}{#2}}

\newcommand{\policy}[1]{\summary{Policy Decision}{#1}}
\newcommand{\policyq}[1]{\summary{Policy Question}{#1}}
\newcommand{\implementation}[1]{\summary{Implementation Decision}{#1}}
\newcommand{\implementationq}[1]{\summary{Implementation Question}{#1}}

\newcommand{\summary}[2]{
	\begin{center}
    \begin{tabular}[h!]{|p{0.8\textwidth}|}
    \hline
    {\bf #1}\\\hline
    #2    \\\hline
    \end{tabular}
\end{center}}

\def\sourcetabsize{2}
\newenvironment{sourcestyle}{\begin{scriptsize}}{\end{scriptsize}}
\def\sourceinput#1{\par\begin{sourcestyle}\verbatimtabinput[\sourcetabsize]{#1}\end{sourcestyle}\par}

\newcommand{\Title}[1]{{\bf #1}}
\newcommand{\Term}[1]{{\em #1}}
\newcommand{\eg}{\textit{e.g.}}
\newcommand{\etc}{\textit{etc.}}
\newcommand{\ie}{\textit{i.e.}}
\newcommand{\etal}{\textit{et al.}}
\newcommand{\Math}[1]{$#1$}
\newcommand{\Path}[1]{\textit{#1}}
\newcommand{\RAID}{RAID}
\newcommand{\url}[1]{\textit{#1}}
\newcommand{\syscall}[1]{\textit{#1}}
\newcommand{\plfs}{PLFS}
\newcommand{\noopfs}{No-opFS}
\newcommand{\upfs}{underlying parallel file system}
\newcommand{\rrz}{Roadrunner test cluster}

\newcommand{\KB}{~KB}
\newcommand{\KBs}{~KB/s}
\newcommand{\Kbs}{~Kbit/s}
\newcommand{\mbs}{~Mbit/s}
\newcommand{\MB}{~MB}
\newcommand{\GB}{~GB}
\newcommand{\MBs}{~MB/s}
\newcommand{\GBs}{~GB/s}
\newcommand{\mus}{\mbox{$\mu s$}}
\newcommand{\ms}{\mbox{$ms$}}

\newcommand{\xaxis}{x-axis}
\newcommand{\yaxis}{y-axis}
\newcommand{\mydefinition}[2]{\item {\bf{#1:}} #2}
\newcommand{\figheight}{1.25in}

\newlength{\captwidth} \setlength{\captwidth}{0.9\textwidth}
\newlength{\captsize}  \let\captsize=\footnotesize

\newcommand{\motivationgraph}[2]{
        \subfloat[#1]{\label{fig:#2-motivation}
           \includegraphics[width=0.25\textwidth]{data/#2/#2-motivation.eps}}
}

\newcommand{\readgraph}[3]{
        \subfloat[#1]{\label{fig:#2-read}
           \includegraphics[width=0.25\textwidth]{#3}}
}

\newcommand{\tablefigure}[2]{
    \subfloat[#1]{\label{eval:#2}\includegraphics[width=0.25\textwidth]
        {data/#2/#2.eps}}
}

% works and makes it smaller but it doesn't put it at the bottom of the page
\newcommand{\myfootnotetext}[1]{
    {\begin{minipage}{\captwidth}
    \let\normalsize=\captsize
    \footnotetext{#1}
    \end{minipage}}
}

%\newcommand{\beforecaption}{\vspace{-.15cm}\begin{spacing}{0.85}}
%\newcommand{\aftercaption}{\hrulefill\vspace{-0.45cm}\end{spacing}}
% \newcommand{\aftercaption}{\vspace{-.45cm}\end{spacing}}
\newcommand{\beforecaption}{\vspace{.2cm}}
\newcommand{\aftercaption}{\vspace{-.2cm}}
\newcommand{\mycaption}[3]{
            \beforecaption
            \let\normalsize=\captsize
            \caption[#2]{\label{#1}{{\small \bf #2}}
            {{\small \em #3}\aftercaption}}
}

%\newcommand{\scribble}[1]{{{\marginpar{\tiny#1}}}}
\newcommand{\scribble}[1]{}

\makeatletter
\AddToShipoutPicture{%
\setlength{\@tempdimb}{.5\paperwidth}%
\setlength{\@tempdimc}{.5\paperheight}%
\setlength{\unitlength}{1pt}%
\put(\strip@pt\@tempdimb,\strip@pt\@tempdimc){%
\makebox(200,-730){{\textcolor[gray]{0.5}%
{\fontsize{.3cm}{.3cm}\selectfont{LANL Technical Release LA-UR 09-02117, \today}}}}
}%
}
\makeatother



\begin{document}

\title{\TITLE}
\if 0
\author{
    John Bent,
    Ryan Kroiss,
    Alfred Torrez,
    Meghan Wingate
    \\
    Los Alamos National Lab
}
\fi
\date{}
\maketitle
\thispagestyle{empty}
\pagestyle{empty}
\section{Preface}

This document attempts to lay out a design path for the planned development
versions of \plfs.  It also attempts to describe how development versions can
move into production and the various issues for transitioning between versions.
Some of the issues are technical and some are policy.  Some of the technical
issues are things like how do we implement directory management in the future
\mds.  Policy issues are questions like whether incompatible layouts are 
handled transparently by \plfs\ or whether they are handled by explicitly
copying data from an old \plfs\ into the new one.  

Our goal is to get \plfs\ into production at LANL.  However, we want to make 
sure that the production releases are bug-free, give performance attractive
to users, and are as easy to use as possible.  Hopefully by spelling out the
issues involved the current version of \plfs\ and the planned versions, we
can identify the target version for our first production release.  There is
of course a tension between releasing something soon and releasing something
good.  We could release the current version today but it has several 
limitations that will be discussed below and we need to determine whether
we are comfortable with those limitations.

At the end of this document, we will also describe current open issues
and other miscellaneous development plans.

\section{Current Version - Version 0}

The current version has support for both \fuse\ and for \adio.  However, the
\fuse\ implementation transparently maps logical paths into physical ones.  It
knows this mapping via a single mount argument naming the physical \store.  To
balance metadata workloads across PanFS director blades, each user's \plfs\
\store\ is located on a different PanFS volume.  This is accomplished via
symbolic links.  For example, \Path{/plfs/ben} is a symbolic link to
\Path{/mnt/plfs/vol5/ben/.plfs\_store}.  When \plfs\ is mounted, it is mounted
on \Path{/mnt/plfs} and directed to use \Path{/panfs/scratch} as its \store.
So a typical name resolution would work as follows:

\begin{enumerate}
\item{user ben opens logical file \Path{/plfs/ben/foo}}
\item{the sym link resolves this to logical path 
        \Path{/mnt/plfs/vol5/ben/.plfs\_store}}
\item{FUSE strips the \Path{/mnt/plfs} and \plfs\ inserts the \store\ path 
        and resolves to physical path 
        \Path{/panfs/scratch/vol5/ben/.plfs\_store}}
\end{enumerate}

This works great for \fuse: the \store\ is hidden from the user.  However, for
\adio, there is no current support for mapping so when a user wants to use
\adio, they need to know and use the path to their \store.  We have observed
that users are confused by the notion of the \store\ and therefore we believe
that the \adio\ in the current version is not ready for a production release. 

\policy{
We need to hide the \store\ from users.  Therefore an \adio\ layer
without a transparent mapping cannot be included in a production release.
}

We can have a production release with just \fuse, but Meghan has observed much
higher bandwidth with the \adio\ layer and believes that the bandwidth with 
just the \fuse\ layer might not be enough to convince users to switch.  

\policyq{Can we release a production version with just \fuse\ and not \adio?}

Recently, the developer of Bulk-IO has improved its bandwidth by tuning its 
IO to match the underlying stripe configuration of PanFS.  Although the
performance of \fuse\ is better than Bulk-IO it is not several times better
whereas the performance of \adio\ is. 

\version{0}{Version 0 has only a \fuse\ layer (\ie\ no \adio\ layer) with a
simple mapping to the \store\ and balances metadata load via symbolic links
external to \plfs\ placing each user on a different volume.}

\section{Shared Mappings - Version 1}

Version 1 will move the mapping logic from \fuse\ into the library so it will
be shared with \adio.  However, we can no longer rely on the mount argument to
declare the \store\ as that is not seen by \adio.  We could use the mount
argument for \fuse\ and use hints for \adio\ but this exposes the \store\ to
users and we have already decided we don't want to do that.  Therefore, we have
decided that we will use an \Path{/etc/plfsrc} file on every node to declare
mappings between logical paths and {\store}s.  This file will be parsed by both
\fuse\ and \adio.

However, one problem is that we can no longer depend on symbolic links to
spread users across volumes.  We can either put all {\store}s on a single
volume or we can put a mapping entry into \plfsrc\ for every user
(\eg\ \Path{/plfs/ben} maps to \Path{/panfs/scratch/vol5/ben/.plfs\_store}).
This might be a maintenance problem however as every time a new user is
added, every \plfsrc\ file will need to be updated.  I don't know
the administrative cost of this.

Once we have both an \adio\ and a \fuse\ layer, we quickly can identify 
many optimizations possible in the \adio\ layer that aren't possible in the
current \fuse\ layer.  [Of course, we could consider extending the \fuse\
layer so that the \plfs\ daemons could communicate with each other but we're
not going to do that.]  These optimizations will be discussed in more detail
in Section~\ref{sec-adio-opt}.  This difference between the \adio\ and \fuse\
layers raises an important policy consideration:

\policyq{Many complicated issues can be easily resolved in the \adio\ layer.
Should we concentrate our efforts there and only support \fuse\ for
functionality and access from the front-ends?  Should we even bother putting
\fuse\ on compute nodes?  Can \fuse\ even work on Cielo?  If it can't work
on Cielo, then we just make inconsistent user environments by putting it on
Roadrunner.}
 
PVFS has decided to concentrate on the ADIO layer and not attempt to provide a
high performance POSIX file system interface.  Perhaps we should follow their
lead. 

\version{1}{Version 1 will have both the \fuse\ and \adio\ layers and a mapping
within the library layer will hide {\store}s from the users.  With a simple
map file, all \plfs\ metadata for all users will go to a single PanFS director.
With a more complicated map, each user's metadata workload will go to different
directors.}

\policyq{Use a simple map and no metadata distribution or a complicated map
which spreads metadata workloads across PanFS directors?  Or skip Version 1
in favor of Version 2?}

\compatibility{Version 1 will be backwards compatible with Version 0.}

\section{Distributed Metadata - Version 2}

A better way to distribute users across volumes is for the \plfsrc\ to list
multiple {\store}s (where each is on a different volume) for each logical path.
Then, instead of relying on symbolic links as in Version~0, or complicated
mapping files as in Version~1, \plfs\ itself will load balance metadata
workloads by hashing every file and placing it on a different \store.  In fact,
this is even better than the crude balancing in Versions~0 and 1 because {\em
every file} is placed on a different volume as opposed to every user. 

In order to do the full \mds\ implementation, we need to decide on mechanisms
for file creation and for directory creation.  

\subsection{File Creation}

File creation seems straight-forward: choose a \store\ and create the file
there.  One open question is how to choose the \store\ on which to create a
particular file.  One approach is to hash the filename, another is to hash the
name of the node.  Although we discuss directory creation in detail below in
Section~\ref{sec-dir}, I want to introduce two basic approaches since the
selection of the directory creation approach influences some of the trade-offs
in the file creation.  One directory creation approach is to eagerly create
each directory on every \store.  Another is to just create each directory on a
single \store\ and then replicate lazily whenever needed.

\implementationq{Place files by hashing file name or node name?}

\begin{itemize}
\item{Place files by hashing on the file name}
\begin{itemize}
\item{Will distribute metadata for
a single client workload.  But this isn't really our target and I doubt the
metadata load from a single client benefits much from distribution anyway.
}
\item{File lookups are easy because we know where to look for the file.}
\item{Intuitively this seems bad for N-1 workloads because all access to the
file will be directed to a single \store.  However, remember that in \plfs\
there are no N-1 workloads (\ie\ each container dropping will be hashed to
a different \store).}
\item{Although every container dropping is hashed to a different \store, the
container itself is hashed to just a single \store\
(if we use lazy directory creation).  Therefore in a large N-1 workload,
every node will try to create the same container in the same location.}
\item{There might be more network contention since every client potentially
interacts with every director}
\end{itemize}
\item{Place files by hashing on the node name}
\begin{itemize}
\item{File lookups must search every \store.} 
\item{Buffering might work better and there might be less network contention
since every client interacts with just a single director} 
\end{itemize}
\end{itemize}

\subsection{Directory Creation}
\label{sec-dir}

First remember that there are two types of directories within \plfs.  There are
the logical directories created by users and then there are the directories
created within \plfs\ containers.  

\implementationq{Should we treat logical directories and directories within
containers in the same way?}

It's tempting to place the directory logic and the file logic and all the
mapping logic in a layer below \plfs.  This means that logical directories
and container directories are handled in the same way.  However, handling
these different types of directories in the same way will probably cause
performance degradations.  If we violate the layering and handle these
differently then the below question of whether to do eager or lazy creation
is easy.  We do eager creation for logical directories and lazy creation
for container directories.

If we treat logical directories and container directories in the same way, then
we have to decide whether to eagerly create an instance of each directory on
every \store\ or whether to lazily create each instance whenever it is needed
to store a child entry.

\implementationq{Eagerly replicate directories across every \store\ or
lazily whenever needed to store a child entry?}

\begin{itemize}
\item{lazy directory creation}
\begin{itemize}
\item{Creating a child entry might require that the parent directory be
created first.  In order to do so, the other {\store}s must be searched
in order to ensure that the parent directory is valid.  This can recurse.  
In fact when Ben and I did a whiteboard
session we saw that this would happen frequently in container creation 
when the files were placed by hashing on the file name.}
\end{itemize}
\item{eager directory creation}
\begin{itemize}
\item{This could be really problematic in an N-1 workload on \fuse\ because 
every client would be trying to create the container on every \store. Most
attempts would fail and I believe a failed \syscall{mkdir} should be faster
than a successful one.}
\item{This could be even more problematic for N-N workloads in both \fuse\
and \adio\ because every client would successfully create a container on
every \store.}
\item{Failures during a \syscall{mkdir} operation could leave the file system
in an inconsistent state.  We probably still require the mechanism that
we create missing parent directories as needed.}
\end{itemize}
\end{itemize}

The more I think about this, the more I believe that:

\implementation{Logical directories should be replicated eagerly and 
container directories should be replicated lazily.}

\version{2}

Version 2 has built-in metadata distribution which works by spreading
files and directories across multiple {\store}s.  Since each store can
be a separate PanFS volume and since each PanFS volume is serviced by 
a different director, this approach distributes the metadata workload 
across multiple metadata servers: an approach that has been heretofore
difficult to achieve in PanFS.

Version 2 will probably replicate logical directories eagerly and
container directories lazily.  Given this, we should re-evaluate the
decision about how to select the \store\ on which to place newly created
files and container directories.  Probably if it wasn't late and I wasn't
tired, I might consider re-ordering these subsections and discussing 
directory creation first.  I feel that treating the logical directories and
container directories differently is the right appraoch although I was
initially reluctant to do it due to increased implementation complexity.

Now that I feel comfortable with that decision, the file placement decision
should be easier.  

\begin{itemize}
\item{Place by hashing on file or container directory path}
\begin{itemize}
\item{Parent dirs might need to be created}
\item{Lookup can go directly to the correct \store}
\item{The set of {\store}s in a map cannot ever change}
\end{itemize}
\item{Place by hashing on nodename}
\begin{itemize}
\item{Parent dirs should always exist}
\item{Lookup must search every \store}
\item{The set of {\store}s in a map can change}
\end{itemize}
\end{itemize}

\compatibility{Version~2 may not be backward compatible.  Older versions
will definitely not be able to read Version~2 containers.  Version~2 may be
able to read containers created by older versions.  If we choose placement
by hashing on nodename, then Version~2 will look for files in all {\store}s.
This will be somewhat unnecessary since all the files will be located in
just a single \store\ but it will work.}

\section{Open issues and miscellaneous plans}

\subsection{Container identification}

Container identification is a pain in the butt but it's very important because 
without it \plfs\ has no way to distinguish between a container and a logical
user-created directory.  We have currently 
identified three possible container identification mechanisms:  

{\bf 1) SUID.}  We can set the SUID bit on the container.  However, PVFS, HDFS,
and possibly other global file systems do not support the SUID bit on 
directories.  We have also seen that some users like to set the SUID bit on
their directories.  If they did this on a directory stored in \plfs, then the
next time they tried to access that directory, \plfs\ would mistakenly export
it as a container.  Also, there is no way to atomically create a directory 
with the SUID bit set.  Therefore in order to use this mechanism, we must
create a temporary directory, set the SUID bit, and then rename it.

{\bf 2) Use a specially named file within the container.}  We can create and
check for a specially named file within the container.  This also runs the risk
that a user creates a directory with an entry that happens to match our
specially named file which would cause \plfs\ to mistakenly export that
directory as a container.  Using a specially named file is the current
mechanism in \plfs~0.1.6 and it is set to be \Path{.plfsaccess113918400} so the
possibility of this accident happening seems unlikely.  This mechanism is also
not atomic and therefore requires a temporary directory which must be renamed.

However, we have recently seen a bug where \plfs\ is failing to create a file;
when it tries to create a dropping in the container, it incorrectly identifies
the container as a logical directory (\ie\ the container identification is
failing).  This might be a \plfs\ bug but right now it appears more likely that
it is a cache coherency problem in PanFS.  One node has created a temporary
directory, created the specially named file within, and has renamed the
temporary directory to the correct container path.  Another node checks that
correct container path and finds a directory {\em without} the specially named
file.  

{\bf 3) Container name mangling.}  A third mechanism would be to mangle the
names of containers (\eg\ when a user creates logical file \Path{foo}, \plfs\
creates a directory named \Path{foo.plfscontainer113918400}).  This has the
advantage that it is atomic and therefore doesn't require temporary
directories.  But the disadvantage is that we need to unmangle these names on
\syscall{readdir}, we need to check for the mangled name on \syscall{mkdir},
and we need to check for both the mangled and unmangled names on
\syscall{stat}.  This can also add weird consistency issues if a logical
directory and a container of the same name are simultaneously created on
different nodes.  Hmmm.  Actually that's a problem with the other mechanisms as
well.  Stupid users.

\subsection{\adio\ optimizations}
\label{sec-adio-opt}

All sorts of stuff about when and whether to flatten and replicate indices.
And questions about how to read indices.  Indices could be globablly
constructed by subsets of the rank reading subsets of the index files and then
broadcasting to create the global index.  

\subsubsection{Gary's ADIO ideas}
Gary's written a bunch about this.
Here is his email (sorry I haven't formatted it nicely yet):

The \plfs\ library needs to support something like this functionality:

\begin{enumerate}
\item{ I have this open fd, please give me the local map for
this fd (this is not for the entire map, it is for my part
only
parms: open fd, pointer that the lib would put an address of some
space the lib mallocs where it puts the map part, and a length of
the map part}
\item{ please store this squashed entire map in this
specified subcontainer
parms: fd, address of map, length of map, container to store}
\item{ mark the metadata for this fd as successfully squashed and distributed
this tells the lib to put a metadata dropping in
parms: fd}

\item{ using this consolidated set of map blobs , make me a squashed map
parms: addr of consolidated map blobs, length of consolidatd map blobs,
addr to put squashed map, length of resultant squashed map}

\item{ is there a metadata dropping that this has been squashed and
distributed for this file and where are the locations one might
look for a replica - this gives a valid list of places where
indexes live and whether or not there is a valid squashed index
(subcontainers),
may need an fd and file path version of this or a flag for lookup type
parms: file path, fd, addr for lib to put location list, length of list}

\item{ open for read with squashed index replica location or a squashed map
provided in memory
parms: open stuff plus replica location or map squashed provided}
\end{enumerate}

So an mpi-io adio app on would do "something like" this:
\\
open for write
\\
write stuff
\\
on close for write "this close for write" is a bit of a complication"
(I include frees required by the lib - other malloc/free would
be required by the adio layer itself)
- all would ask for their fd (local) index blob  (new func 1)
 (I suppose we didnt write anything is a possible outcome)
\\
- all reduce these local index blobs to proc 0
\\
- proc 0 calls using these parts make me a squashed map (new func 4)
\\
- proc 0 calls is there squashed metadata for this file
  which gives proc 0 a list of subcontainers (new func 5)
\\
- proc 0 assigns work (squashed map and subcontainer) to get
  all the squashed maps distributed
\\
- all workers call please store this squashed map in this subcontainer
(new func 2)
\\
- all free the blobs local blobs (allocated by the lib)
\\
- proc 0 frees squashed map (allocated by the lib)
\\
- all but proc 0 barrier
\\
- proc 0 mark metadata for this fd as successfully squashed and
replicated  (new func 3)
\\
- proc 0 barrier
\\
- all call close

So you can see that the lib doesnt do any mpi magic and the adio (or
app) doesnt know what is in the map or a subcontainer, they are just
blobs of data used to distribute, because it has the distribution
mechanism.

You notice that the distribution of the squashed indexes to
the brethren does not just blindly do a broadcast and write
the index, because you might want to write fewer indexes
than procs.

So this gets the index squashed on close for write by adio.
We might want a hint on whether you want squashed indexes
on this mpi run so you could avoid this if you dont want/need
it.  Probably should plan on that up front even if we dont
want to use it.

Now the adio on open for read would
proc 0 in the adio layer would check to see if there
is there a valid squashed index and get the location list  (new func 5)
\\
--if valid squashed index
\\
----  assign replicas to the brethren
\\
---- everyone opens using that squashed index  (few func 6 using
subcontainer)
\\
--if there is no valid squashed index
\\
---- proc 0 has a list of places where indexes live - from the query for
squash above - even though there is no squashed index,
\\
---- proc zero assigns work to the brethren (may not be equal number of
workers and number of indexes to read),
\\
---- each proc does open read with their index part assigned (new func 6
using subcontainer)
\\
---- get local map (new func 1)
\\
---- send it back to proc 0,
\\
---- when proc 0 gets them all it calls build me a squashed index from
these parts into the lib (new func 4)
\\
---- broadcast resultant squashed map to all,
\\
---- everyone opens with squashed map provided (new func 6 - using mem
map) (I suppose you could write the squashed maps to disk here but that
seems bad on a read op)

adio close on read has nothing much to do maybe frees or something,
again you could write the squashed indexes here but on a read
that is probably bad

A stand alone mpi utility to squash would then:
\\
- just be an mpi program that open rw (because it is open for read it
would efficiently get the squashed map with as many procs as you have
available - doesnt have to match the number of containers) - everyone
would end up with a copy of the fully squashed map
\\
- close for rw  (Because this is a close for write the squashed indexes
are written)

A stand alone non mpi utility to squash would be essentially the same
thing as what the mpi open for rw and close for write is, but
the farm out the work would be a loop to do the work and not to assign
the work, or maybe do it with threads where there was procs, which
would make the stand alone utility to be basically the same as the mpi
utility.

Again, this close for read/close for write seems to be a slight
complication, but maybe an important capability.

Of course this is all notional, but

\begin{enumerate}
\item{this seems doable}
\item{works where num subcontainers doesnt equal num procs}
\item{leverages mpi flie open and close}
\item{gets us easy stand alone mpi and non mpi squasher}
\item{nearly optimal for mpi and helps posix hugely}
\end{enumerate}

Now the posix open for read does something similar to
what proc zero and the other procs do, it would need
to call the same sorts of routines.  It would probably
just need to randomly pick a subcontainer to
do the open with container for squashed index.
The posix close probably wouldnt do a squash, as it
wont know when people are done and how to do it and
who else is doing it etc.



\section{Final thoughts}

I feel like I can identify a path forward.  Deciding to eagerly replicate
logical directories and lazily replicate container directories hugely
simplifies things.  However, we still need to decide 

\begin{itemize}
\item{whether to place files and 
container directories by hashing their paths or by hashing the node names.  
}
\item{whether to attempt to optimize both the \adio\ and \fuse\ layers or
just \adio}
\item{whether to run \fuse\ on compute nodes or just front-ends}
\item{which version to push for production}
\end{itemize}


\end{document}
