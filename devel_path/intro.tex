\section{Introduction}
Experiment management in any domain is challenging.  There is a perpetual
feedback loop cycling through planning, execution, measurement, and analysis.
The lifetime of a particular experiment can be limited to a single cycle 
although many require myriad more cycles before definite results can be 
obtained.  Within each cycle, a large number of \Term{\subs} may be executed
in order to measure the effects of one or more independent variables.  

Experiment management in high performance computing (HPC) follows this general
pattern but also has three unique characteristics.  One, computational science
applications running on large supercomputers must deal with frequent platform
failures which can interrupt, perturb, or terminate running experiments.  Two,
these applications typically integrate in parallel using MPI as their
communication medium; each \sub\ within HPC is typically executed with an
\Term{mpirun} command.  Three, there is typically a scheduling system (\eg\
Condor, Moab, SGE, \etc) acting as a gate-keeper for the HPC resources.

In this paper, we introduce \namefull\ (\name), an experimental management
framework simplifying all four phases of experiment management.  \name\
simplifies experiment planning by allowing the user to describe their
experimental goals without having to fully construct the individual parameters
for each task.  To simplify execution, \name\ dispatches the \subs\ itself
thereby freeing the user from remembering the often arcane methods for
interacting with the various scheduling systems.  By providing
\Term{transducers}, \name\ automatically measures and records important
information about each \sub; these transducers can easily be extended to
collect additional measurements specific to each experiment.  Transducers are
implemented as wrappers that execute each \sub\ and can be extended to parse
the output of each \sub;  the term is borrowed from a similar concept in
Semantic File Systems~\cite{gifford-sfs}.  In both \name\ and Semantic File
Systems, transducers automatically generate semantically meaningful information
in \kv\ pairs which can then be easily searched.  Finally, experiment analysis
is simplified by providing \dbviz\, a general database visualization framework
that allows users to quickly and easily interact with their measured data.
\dbviz\ provides two important advanced features for improving data analysis:
\Term{outlier analysis} and \Term{hidden difference search}.  A typical
experiment workflow is shown in Figure~\ref{typical}; Figure~\ref{lem-plan}
shows the same workflow augmented with the various components of \name.

Although our experiment management framework has been designed for use at Los
Alamos National Lab (LANL) and therefore generally for an HPC environment, we
believe it can be easily modified to be suitable for experiment management in a
Grid environment as well.  Throughout this paper, we discuss how \name\ reduces
the complexity of managing experiments that may span multiple supercomputers
and the different scheduling systems on each.  In terms of experiment
management, supercomputers in HPC are very similar to clusters in the Grid.
Although the Grid introduces additional complexity due to multiple
administrative domains, we believe that \name\ can be useful in this
environment as well but this is not a focus of this paper.

\if 0
% here's a way to try to make it just be one column wide but it didn't
% work for me, it made it one columnwide but it moved it to the end of the
%document
\begin{wrapfigure}{r}{0.5\textwidth}[tb]
    \begin{center}
        \psfig{file=figs/architecture.eps,width=0.48\textwidth}
    \end{center}
% we can also try to use SCfigure for captions on the side, I tried that
% and it looks sort of weird.  But I'm also not happy with the default
% which makes it way too big and leaves too much white space..
\begin{SCfigure}
    \centering
    \includegraphics[width=0.5\textwidth]{figs/architecture.eps}
\fi

%\singlespace
\begin{figure*}[tb]
    \centering
    \includegraphics[width=0.7\textwidth]{figs/architecture.eps}
    \mycaption{fig-arch}{Data Reorganization.}{
This figure depicts how PLFS reorganizes an N-1 strided checkpoint file onto
the \upfs.  A parallel application consisting of six
processes on three compute nodes is represented by the top three boxes.  Each 
box represents a compute node, a circle is a process, and the three small boxes
below each process represent the state of that process. The processes
create a new file on \plfs\ called \textit{checkpoint1}, causing  \plfs\ in
turn to
create a \Term{container} structure on the underlying parallel file system.
The container consists of a top-level directory also called
\textit{checkpoint1} and several
sub-directories to store the application's data.  For each
process opening the file, \plfs\ creates a \Term{data file} within one of the
sub-directories, it also creates one \Term{index file} within that same 
sub-directory 
which is shared by all processes on a compute node.
For each write, \plfs\ appends the data to the corresponding data file 
and appends a record into the appropriate index file.  This record
contains the length of the write, its logical offset, and a pointer
to its physical offset within the data file to which it was appended.
To satisfy reads, \plfs\ aggregates these index files to create a lookup
table for the logical file.  Also shown in this figure are the
\textit{access} file, which is used to store ownership and privilege
information about the logical file, and the \textit{openhosts} and
\textit{metadata} sub-directories which are used to cache metadata in order
to improve query time (\eg\ a \syscall{stat} call). 
}
\end{figure*}
%\doublespace


The remainder of the paper is organized as follows.  We describe our design in
Section~\ref{arch}.  In Section~\ref{eval}, we offer a demonstration of our
usage model.  We provide a comparison to related work in Section~\ref{related},
current status and future work in Section~\ref{future}, and a conclusion in
Section~\ref{conclude}.
