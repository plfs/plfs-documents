\section{Conclusion}
\label{conclude}

Experiment management is a cumbersome process.  Users must plan, execute,
monitor, and analyze multiple \subs.  This process is often ad-hoc and
each new experiment requires the development of new experiment management
procedures.  In computational science, and High Performance Computing in
particular, developing a new experiment often requires that multiple new
scripts are written: some to generate multiple \subs, some to execute
them, and some to parse their outputs to enable analysis. 

\name\ simplifies this process by providing a general and modular framework
by which users can describe their experiments and \name\ will automatically
generate \subs, execute them, and collect the resulting data.  Additionally,
\name\ provides a simple interface to augment the data collection and to
monitor and replay experiments.  A visualization component allows interactive
data exploration greatly simplifying analysis and experiment tuning.  
Production use at Los Alamos National Labs has demonstrated the value of \name\
in enabling end-to-end experiment management in HPC.
