\section{Related Work}
\label{related}

\name\ introduces a new way to manage experiments in a HPC environment, greatly
simplifying the experimental process.  \name\ submits and executes jobs, parses
the results, inserts them into a database, and allows the user to graphically
and interactively view the results. Similar efforts can be found, but none
offer the holistic approach of \name\ nor its HPC focus.

GridBot~\cite{gridbot} is a system for executing experiments across multiple
grids and clusters. Its high-level goal is very similar to \name; both systems
are designed to ease the burden of computational scientists to manage very
large experiments with many \subs\ (\subs\ in GridBot are called \Term{Bags of
Tasks}). GridBot's focus however is on dispatch; its dispatch engine is
designed for the Grid however where the dispatcher in \name\ is designed for an
HPC environment.  Focused on dispatch, GridBot lacks any analogues for
transducers, \dbviz, or the ability to monitor and replay experiments.
Ultimately, the two systems are highly complementary: \name\ provides
end-to-end experiment management for HPC; combining it with the grid-aware
dispatching in GridBot would enable end-to-end experiment management for the
Grid.  

VGrADS~\cite{vgrads} is a similar approach that provides a virtualized Grid
execution system; like GridBot, it is geared for the Grid whereas \name\ is
geared for HPC. VGrADS also attempts to provide reliable quality of service and
fault tolerance using automated task replication.  These features complement
\name's more holistic approach.  

Zoo~\cite{livny-zoo} is another experiment management framework for desktop
computing.  Similar in concept to \name\, it acknowledges the experiment
feedback loop and provides a framework for making the experimental
process more manageable.  Zoo has modules, such as Opossum and Frog, for each step of the process, from designing and setting up experiments through data
collection and analysis.  Although novel for its time, Zoo, and its focus on
the desktop, provides insufficient management for the much more complex HPC
environment.  In addition, Zoo lacks the advanced features of \dbviz, such
as outlier analysis and hidden difference search.

GridDB~\cite{griddb} provides a data-centric overlay to process-centric grid
middleware.  The key differences between \name\ and GridDB exist in how they
handle data processing and analysis.  GridDB uses a declarative interface and
type checking to deal with data processing and interactive query processing to
support analysis.  These techniques offer flexibility, but they also make the
data more abstract.  \name\ uses transducers to convert experimental results
into usable data.  Transducers are slightly less flexible, but they help to
maintain the raw form of the data.  \dbviz\ offers easier data analysis and
\name\ seeks to keep data processing and analysis more decoupled by using
transducers and \dbviz.  \name\ provides a larger end-to-end focus whereas
GridDB offers a greater emphasis on automated computational steering.
