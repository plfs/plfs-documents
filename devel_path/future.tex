\section{Current Status and Future Work}
\label{future}

\name\ is used primarily by system integrators at Los Alamos National Lab to
monitor several parallel storage systems on LANL supercomputers.  It has been a
valuable tool enabling much easier analysis of the impact of configuration
changes and software upgrades on these systems.  The ease by which experiments
can be replayed allows significantly improved consistency of experiments
enabling much cleaner \Term{apples-to-apples} comparisons from before and after
configuration changes.  \dbviz\ and the central SQL server simplify cross
system comparisons that were previously extremely manually intensive.
Additionally, \name\ is increasingly being adopted by other groups at LANL and
has been used by several vendors to benchmark new systems.  

We are working with one code team in particular at LANL that has current
workflow inefficiencies in their regression testing that could be drastically
improved using \name.  They currently use an ad-hoc set of tools that has some
of \name's functionality but it is not well consolidated nor organized and
other functionality is entirely absent.  This team runs nightly, weekly, and
monthly tests where various metrics such as CPU utilization and I/O bandwidth
are measured.  The data is held in a database but the team lacks the capability
to visualize the data interactively.  When measured values fall outside of
predetermined thresholds, the team is alerted to this but lacks an easy method
to identify a root cause.  Interactive sessions with \dbviz\ would enable the
team to quickly isolate, identify, and analyze the particular \subs\ that fell
outside of the thresholds.  

Their static visualization tools are very lacking in this regard; we have seen
instances in which they have missed tests of interest because their predefined
graphing utilities left some \subs\ literally off the graphs.  Also, they have
been plagued by an abundance of data which overwhelms their graphs with noise;
\dbviz\ can allow them to interactively filter and focus their analysis.
Finally, even when they have ultimately identified root cause for performance
anomalies, they lack utilities to rerun just those \subs.  The replay ability
of \name\ to apply a query to fetch a particular set of \subs\ to resubmit will
drastically improve their workflow and hasten their analysis of performance
anomalies. 
 
\dbviz\ is publicly available through SourceForge; the other components of
\name\ will also be made publicly available pending the copyright process.

We are continuing to improve \name\ through the development of additional
features.  In the future, we aim to add a web-based interface to centralize
experiment management and more easily allow multiple users to interact with
collaborative experiments.  Additionally, we plan to modularize the database
integration to remove the SQL-only restriction. 
