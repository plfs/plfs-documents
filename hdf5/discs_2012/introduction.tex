Hierarchical Data Format (HDF5) is a data model~\cite{hdf5_ad11}, library and file format for storing and managing data. It is designed for flexible and efficient I/O. HDF5 defines an information set, which is a container of array variables, groups, and types. The data model defines mechanisms for creating associations between various information items. HDF5 is widely used in the industry and scientic domain, in understanding global climate change, special effects in film production, DNA analysis, weather prediction, financial data management etc. (citation here)
Parallel HDF5 (PHDF5) enables developing high performance, parallel applications using standard technologies like MPI in conjunction with HDF5. An HDF5 file created using PHDF5 is compatible with serial HDF5 files and is shareable between different platforms. PHDF5 exports a standard parallel I/O interface which then uses MPI's parallel I/O functionality. 

The central component of HDF5, the file, is a self-describing format which combines data and metadata. Users typically store multiple HDF5 objects in a single file, and the library stores metadata alongwith data that describes the relationships between various objects amongst other metadata. 
However, this native single-file format has its disadvantages. A PHDF5 application has multiple processes accessing a single HDF5 file. Many popular parallel file systems are known to behave poorly under these circumstances. Secondly, since many HDF5 objects are stored in a single file, this eliminates any possibility of performing useful semantic analysis on objects beyond the scope of an HDF5 application. In this paper, we address these shortcomings of HDF5, viz. performance issues due to multiple processes accessing a single shared file, and lack of a way to perform useful post-processing on HDF5 objects inherent due to the native file format. 

PLFS is a middleware virtual file system developed at Los Alamos National Lab (LANL). It converts writes to a shared logical file into writes to multiple physical files.  
Thus, it interposes on application I/O and converts its I/O pattern into one that is more suitable for the underlying parallel file system.
PLFS is popularly used as a checkpoint file system where applications demonstrating heavy checkpoint workloads are known to benefit significantly from PLFS.

In this work, we have developed a new plugin for HDF5 using its recently introduced Virtual Object Layer (VOL). The VOL exports an interface that allows writing plugins for HDF5, where it primarily passes references to HDF5 objects, thereby enabling developers to store objects in a format different from the default HDF5 file format. 
We have written a plugin with two main objectives. The plugin stores data in a unique way that enables semantic post-processing on HDF5 objects outside the scope of the HDF5 application. Secondly, PLFS converts N-->1 accesses into N-->N accesses, thereby showing a significant improvement in performance as well. Preliminary results using HDF5's h5perf performance tool show a Xx performance improvement in I/O overall traditional PHDF5.

The rest of the paper is organized as follows. We represent details about HDF5 and PLFS in section 2, describe our plugin design and implementation in section 3, and our evaluation in section 4. We present related work in section 5, the current status of the plugin alongwith future work in section 6, and finally we conclude in section 7.

