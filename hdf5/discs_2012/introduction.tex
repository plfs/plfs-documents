Hierarchical Data Format (HDF5) is a technology suite for efficient management of large and complex data~\cite{hdf5_ad11}. It supports complex relationships between data and dependencies between objects. HDF5 is widely used in industry and scientic domains, in understanding global climate change, special effects in film production, DNA analysis, weather prediction, financial data management etc.~\cite{hdf5_users}
Parallel HDF5 (PHDF5)~\cite{phdf5} enables developing high performance, parallel applications using standard technologies like MPI in conjunction with HDF5. 
%An HDF5 file created using PHDF5 is compatible with serial HDF5 files and is shareable between different platforms. 
PHDF5 exports a standard parallel I/O interface which itself uses MPI's parallel I/O functionality. This is used alongwith parallel file systems to achieve high performance I/O.

An HDF5 file is a self-describing format which combines data and metadata. It is a container in which users typically store multiple HDF5 objects alongside their metadata. 
However, this native single-file format has its disadvantages. A PHDF5 application has multiple processes accessing a single file (i.e. N-1 access pattern). Unfortunately, many popular parallel file systems are known to behave poorly under these circumstances~\cite{lustre_perf}\cite{nersc_io}. Secondly, since many HDF5 objects are stored in a single file, this complicates the possibility of performing useful semantic operations on individual objects. In this paper, we enhance the performance of HDF5 by addressing the performance issues inherently arising from poor file system performance on N-1 access patterns, and augment performing  useful post-processing on HDF5 objects. 

PLFS (Parallel Log-Structured File System) is a middleware virtual file system developed at Los Alamos National Lab (LANL)~\cite{plfs_sc09}. It converts writes to a shared logical file into writes to multiple physical files to overcome the performance bottleneck associated with N-1 writes. However, it suffers from an inability to understand the structure of the data that it stores.

In this work, we have developed a new plugin for HDF5 using its recently introduced Virtual Object Layer (VOL).  
This plugin serves two purposes: it stores data in a unique way that enables semantic post-processing on HDF5 objects, and by using PLFS to convert N-1 accesses into N-N accesses, thereby resulting in improved I/O performance.

%The rest of the paper is organized as follows. We present some background on HDF5 and PLFS in section 2, our plugin design and implementation in section 3, our evaluation in section 4, and our conclusion in section 5.

