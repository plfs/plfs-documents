\documentclass[10pt]{article}
\usepackage{fullpage}
\usepackage[pdftex]{graphics}
\usepackage{enumitem}
\usepackage{subfig}
\usepackage{amsmath}
\usepackage{alltt}
\setlist{noitemsep,topsep=0pt}
\setitemize{noitemsep,topsep=0pt}
\setenumerate{noitemsep,topsep=0pt} 

\addtolength{\textwidth}{.5in}
\addtolength{\hoffset}{-.25in}
\addtolength{\textheight}{.5in}
\addtolength{\voffset}{-.5in}

% macros to use in the text
\newcommand{\TITLE}{PLFS Roadmap}
\newcommand{\mds}{metadata distribution layer}
\newcommand{\fuse}{PLFS-FUSE}
\newcommand{\adio}{PLFS-ADIO}
\newcommand{\store}{back-end store}
\newcommand{\plfsrc}{\Path{/etc/plfsrc}}

%\newcommand{\summary}[1]{\framebox[6in]{#1}}

\newcommand{\compatibility}[1]{\subsection{Compatibility}{#1}}
\newcommand{\version}[2]{\subsection{Version #1 Summary}{#2}}

\newcommand{\policy}[1]{\summary{Policy Decision}{#1}}
\newcommand{\policyq}[1]{\summary{Policy Question}{#1}}
\newcommand{\implementation}[1]{\summary{Implementation Decision}{#1}}
\newcommand{\implementationq}[1]{\summary{Implementation Question}{#1}}

\newcommand{\summary}[2]{
	\begin{center}
    \begin{tabular}[h!]{|p{0.8\textwidth}|}
    \hline
    {\bf #1}\\\hline
    #2    \\\hline
    \end{tabular}
\end{center}}

\def\sourcetabsize{2}
\newenvironment{sourcestyle}{\begin{scriptsize}}{\end{scriptsize}}
\def\sourceinput#1{\par\begin{sourcestyle}\verbatimtabinput[\sourcetabsize]{#1}\end{sourcestyle}\par}

\newcommand{\Title}[1]{{\bf #1}}
\newcommand{\Term}[1]{{\em #1}}
\newcommand{\eg}{\textit{e.g.}}
\newcommand{\etc}{\textit{etc.}}
\newcommand{\ie}{\textit{i.e.}}
\newcommand{\etal}{\textit{et al.}}
\newcommand{\Math}[1]{$#1$}
\newcommand{\Path}[1]{\textit{#1}}
\newcommand{\RAID}{RAID}
\newcommand{\url}[1]{\textit{#1}}
\newcommand{\syscall}[1]{\textit{#1}}
\newcommand{\plfs}{PLFS}
\newcommand{\noopfs}{No-opFS}
\newcommand{\upfs}{underlying parallel file system}
\newcommand{\rrz}{Roadrunner test cluster}

\newcommand{\KB}{~KB}
\newcommand{\KBs}{~KB/s}
\newcommand{\Kbs}{~Kbit/s}
\newcommand{\mbs}{~Mbit/s}
\newcommand{\MB}{~MB}
\newcommand{\GB}{~GB}
\newcommand{\MBs}{~MB/s}
\newcommand{\GBs}{~GB/s}
\newcommand{\mus}{\mbox{$\mu s$}}
\newcommand{\ms}{\mbox{$ms$}}

\newcommand{\xaxis}{x-axis}
\newcommand{\yaxis}{y-axis}
\newcommand{\mydefinition}[2]{\item {\bf{#1:}} #2}
\newcommand{\figheight}{1.25in}

\newlength{\captwidth} \setlength{\captwidth}{0.9\textwidth}
\newlength{\captsize}  \let\captsize=\footnotesize

\newcommand{\motivationgraph}[2]{
        \subfloat[#1]{\label{fig:#2-motivation}
           \includegraphics[width=0.25\textwidth]{data/#2/#2-motivation.eps}}
}

\newcommand{\readgraph}[3]{
        \subfloat[#1]{\label{fig:#2-read}
           \includegraphics[width=0.25\textwidth]{#3}}
}

\newcommand{\tablefigure}[2]{
    \subfloat[#1]{\label{eval:#2}\includegraphics[width=0.25\textwidth]
        {data/#2/#2.eps}}
}

% works and makes it smaller but it doesn't put it at the bottom of the page
\newcommand{\myfootnotetext}[1]{
    {\begin{minipage}{\captwidth}
    \let\normalsize=\captsize
    \footnotetext{#1}
    \end{minipage}}
}

%\newcommand{\beforecaption}{\vspace{-.15cm}\begin{spacing}{0.85}}
%\newcommand{\aftercaption}{\hrulefill\vspace{-0.45cm}\end{spacing}}
% \newcommand{\aftercaption}{\vspace{-.45cm}\end{spacing}}
\newcommand{\beforecaption}{\vspace{.2cm}}
\newcommand{\aftercaption}{\vspace{-.2cm}}
\newcommand{\mycaption}[3]{
            \beforecaption
            \let\normalsize=\captsize
            \caption[#2]{\label{#1}{{\small \bf #2}}
            {{\small \em #3}\aftercaption}}
}

%\newcommand{\scribble}[1]{{{\marginpar{\tiny#1}}}}
\newcommand{\scribble}[1]{}

%\makeatletter
\AddToShipoutPicture{%
\setlength{\@tempdimb}{.5\paperwidth}%
\setlength{\@tempdimc}{.5\paperheight}%
\setlength{\unitlength}{1pt}%
\put(\strip@pt\@tempdimb,\strip@pt\@tempdimc){%
\makebox(200,-730){{\textcolor[gray]{0.5}%
{\fontsize{.3cm}{.3cm}\selectfont{LANL Technical Release LA-UR 09-02117, \today}}}}
}%
}
\makeatother



\begin{document}

\vspace{-1in}
\title{PLFS Policy Regarding Writes by Others}
\if 0
\author{
    John Bent,
    Ryan Kroiss,
    Alfred Torrez,
    Meghan Wingate
    \\
    Los Alamos National Lab
}
\fi
\date{}
\maketitle
\thispagestyle{empty}
\pagestyle{empty}
\vspace{-1.5cm}

PLFS files are physically stored in a directory structure.  This creates some
difficulty since the UNIX permissions work differently on directories than they
do on files.  We would like to access the security protections of the 
underlying file system (\store) as much as possible but we might occasionally
have to attempt some enforcement ourselves.  Of course, this opens a security 
hole since the physical files on the \store\ cannot be hidden.  

One particular thorny problem is allowing users to grant write permissions to
other users on their files (writes by others).  We can disallow this but if
we'd like to export a file system with capabilities that users are accustomed
to, than we might try to support this.  A state diagram depicting possible
policy decisions is shown in Figure~\ref{fig-state}.

Our first option is to just disallow writes by others.  This is easy
to implement and will not allow any unauthorized access on the \store.  To
implement this, we merely remove write permissions for the group and
for others ({\tt \ie\ mode \& (!(S\_IWGRP|S\_IWOTH))})

\begin{figure}[h]
\centering
\includegraphics{state.png}
\mycaption{fig-state}{Possible Policies in Regards to Writes by Others.}{This 
state diagram depicts the different policies and their implications in regards
to whether and how PLFS can support owners of files granting write permissions
on those files to other users or groups.}
\end{figure}

If we want to allow writes by others, there are two ways in which we can 
implement this.  The first way is that when \droppings\ are created inside
\plfs\ {\container}s, they are created with permissions as defined by the 
{\tt mode\_t} passed to the {\tt open()} call.  The problem with this 
approach is that when user B writes to A's file, \droppings\ will be created
in A's container with permissions that may not allow A access to them.
This could prevent A from unlinking the file or potentially even reading the
file if B's {\tt mode\_t} is very restrictive.  To somewhat ameliorate this
problem, when A later does a {\tt chmod()} or a {\tt chown()} on the file 
through \fuse\ when \fuse\ is run as root, \fuse\ will recurse through the
file and restore ownership of all \droppings\ to A.  So A can eventually
have full access to the file after doing a {\tt chmod()} or {\tt chown()}.

The third approach, and the approach taken by Version 1.0.0, is for all
\droppings\ to always be created with full permissions (\ie\ ({\\tt
 (S\_IRWXU|S\_IRWXG|S\_IRWXO)}) which allows A to easily remove any 
\droppings\ created by B.  We then attempt to limit permissions by
setting the permissions of all directories within the \container\ 
appropriately.  The drawback of this approach is that if A gives B read
permissions on the file, then the \droppings\ will be globally writable
and the directories will allow B read access.  Therefore B cannot 
create new droppings but can overwrite and truncate existing droppings.
We explicitly check for this and prevent this for access through \fuse,
but this does allow B unauthorized access through the \store.

Given these constraints and our desire to make a filesystem conforming
the most closely to user expectations and given the lack of obvious
visibility to the \store, we have currently chosen to use the third
approach in Version 1.0.0.  This allows write by others and allows
the original owner full access to their files after write by others
without any intermediate steps such as {\tt chmod()} or {\tt chown()}.
This does allow unauthorized overwriting on the \store\ when an other
has only read access.

\end{document}
