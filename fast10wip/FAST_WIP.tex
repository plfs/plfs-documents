\documentclass[letterpaper,10pt]{article}
%\pagestyle{empty}
\usepackage{graphicx}
\usepackage{verbatim}
\usepackage{authblk}
\usepackage[top=0.75in, bottom=0.75in, left=0.75in, right=0.75in]{geometry}

%\setlength\textheight{9.0in}


\title{Enabling Scientific Application I/O on Cloud FileSystems}

\author[1]{Milo Polte (student)}
\author[2]{Esteban Molina-Estolano (student)}
\author[3]{John Bent}
\author[2]{Scott Brandt}
\author[1]{Garth Gibson}
\author[4]{Maya Gokhale}
\author[2]{Carlos Maltzahn}
\author[3]{Meghan Wingate}
\affil[1]{Carnegie Mellon University}
\affil[2]{UC Santa Cruz}
\affil[3]{Los Alamos National Laboratory}
\affil[4]{Lawrence Livermore National Laboratory}

\date{}


\begin{document}

\bibliographystyle{plain}

\maketitle

    Today, the storage of many of the world's largest computing
    clusters is being supported not through traditional parallel
    filesystems, such as Lustre\cite{schwan:linux03}, the Parallel Virtual
    Filesystem (PVFS)\cite{carns:PVFS}, or PanFS\cite{nagle:sc04}, but
    through ``cloud'' filesystems such as GoogleFS\cite{ghemawat:gfs},
    the Hadoop Distributed Filesystem (HDFS)\cite{hdfs}, and the
    Amazon Simple Storage Service (Amazon S3).\cite{amazons3} Although
    these filesystems provide appropriate performance and resilience
    for web apps, their APIs are often unsuitable for the I/O patterns
    of a scientific application. Scientific applications typically use
    a POSIX interface, often with additional support for semantics
    such as concurrent writers or out of order writes. These
    applications cannot presently run at all on a cloud filesystem,
    since these filesystems lack the necessary semantics. We would
    like to remedy this deficiency: running scientific applications on
    cloud filesystems would allow sites with investments in cloud
    filesystems to leverage existing hardware to run scientific
    computing applications as well.
    

    Previously, the authors have designed a thin interposition layer,
    the Parallel Log-structured Filesystem (PLFS)\cite{plfs}, to
    improve the I/O speed of scientific applications running on a
    parallel filesystem. PLFS transparently decouples inefficient,
    concurrent writers accessing a shared file into efficient,
    log-style writers maintaining individual data files, while still
    providing the application with a view of a single, flat file.

    Although this previously work was motivated by performance rather
    than enabling functionality, we believe the same techniques can be
    adapted to support concurrent writers in a cloud filesystem. In
    this work, we will extend PLFS to run on the Hadoop filesystem and
    add shared write semantics to it. This will enable scientific
    applications, such as simulation checkpointing, to run unmodified
    on top of the Hadoop filesystem. We plan to compare the
    performance against the same scientific apps running on PVFS (a
    common parallel filesystem used at National Labs) on the same
    hardware and evaluate the trade-offs in performance. 


\bibliography{references}

\end{document}
